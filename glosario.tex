\chapter{Glosario}\label{cap:glosario}
\begin{description}
\item[ACPI] \texttt{Advanced Configuration and Power Interface}. Especificaci\'on como est\'andar abierto para la configuraci\'on de dispositivos y manejo de energ\'ia por parte del sistema operativo.
\item[AMD-V] \texttt{Advanced Micro Devices Virtualization}. Conjunto de extensiones implementadas por AMD en sus procesadores para soportar la virtualizaci\'on hardware.
\item[API] \texttt{Application Programming Interface}. Protocolo cuya finalidad es ser usado como interfaz entre los distintos componentes de un software para comunicarse.
\item[APIC] \texttt{Advanced Programmable Interrupt Controller}. Dispositivo utilizado para combinar distintas fuentes de se\~nales de interrupci\'on, en una o m\'as l\'ineas de CPU, permiendo asignar distintos niveles de prio\-ri\-dad a cada interrupci\'on. As\'i, cuando un dispositivo tenga que gestionar interrupciones externas, lo har\'a siguiendo el nivel de prioridad que acarree.
\item[AWS] \texttt{Amazon Web Services}. Conjunto de servicios web ofrecidos por Amazon, Inc. que permiten la explotaci\'on remota y bajo demanda de su infraestructura computacional.
\item[Cableado] En Ingenier\'ia del Software el t\'ermino \emph{cableado}, tomado del ingl\'es \emph{hardcoded}, hace referencia a aquel t\'ermino de entrada o de configuraci\'on de una aplicaci\'on que est\'a incluido en el c\'odigo fuente.
\item[CLI] \texttt{Command Line Interface}. Hace referencia a toda aplicaci\'on o he\-rra\-mien\-ta cuya interfaz de manipulaci\'on es la l\'inea de comandos de un terminal.
\item[CUDA] \texttt{Compute Unified Device Architecture}. Plataforma de com\-pu\-ta\-ci\'on paralela y modelo de programaci\'on creado por NVIDIA e implementado por las GPUs que produce. Su importancia radica en que, usan\-do CUDA, las GPUs pueden utilizarse para ejecutar computaciones similares a las soportadas por CPUs tradicionales.
\item[DFS] \texttt{Distributed File System}. Cualquier sistema de ficheros almacenado en una red de computadores y accesible de forma concurrente por m\'ultiples usuarios desde distintos puntos.
\item[DHCP] \texttt{Dynamic Host Configuration Protocol}. Protocolo de red utilizado para la configuraci\'on de acceso a una red cualquiera de forma autom\'atica.
\item[DNS] \texttt{Domain Name Server}. Sistema de nombres distribuido en red que asocia cierta informaci\'on de los dispositivos de un dominio. M\'as com\'unmente, traduce los nombres de un dominio, con significado para los usuarios, en direcciones IP requeridas para la localizaci\'on de un recurso de red.
\item[Amazon EC2] \texttt{Elastic Compute Cloud}. Servicio web de Amazon, Inc. que permite el aprovisionamiento de infraestructura computacional bajo demanda.
\item[EPT] \texttt{Extended Page Tables}. Sistema de virtualizaci\'on hardware de Intel, Inc. que permite al software hu\'esped modificar sus propias tablas de p\'aginas de memoria y manejar directamente los fallos de p\'agina. Esto evita que el \emph{VMM} gestione las salidas de la m\'aquina virtual, asociadas a la virtualizaci\'on de la tabla de p\'aginas, que tanto ralentizan la ejecuci\'on virtualizada sin EPT.
\item[Funci\'on Hash] Cualquier algoritmo o subrutina que asocia un gran espacio de datos de longitud variable, a otro espacio m\'as peque\~no y de longitud fija.
\item[HTTP] \texttt{HyperText Transfer Protocol}. Protocolo de transferencia de da\-tos est\'andar entre navegadores web y servidores en Internet.
\item[HVM] \texttt{Hardware Virtual Machine}. Aproximaci\'on de implementaci\'on de virtualizaci\'on que posibilita explotar de forma eficaz un anfitri\'on, con ayuda de sus componentes hardware, normalmente CPU, para soportar la ejecuci\'on de plataformas virtuales. Tambi\'en conocido como \emph{vir\-tua\-li\-za\-ci\'on nativa}.
\item[iptables] Herramienta administrativa y servicio asociado para el filtrado de paqueter\'ia IPv4 y \emph{NAT}.
\item[IT] \texttt{Information Technology}. Engloba cualquier concepto relacionado con tecnolog\'ia de la informaci\'on, como \emph{redes}, hardware, software, Internet, etc. as\'i, como las personas que trabajan con esas tecnolog\'ias.
\item[JRE] \texttt{Java Runtime Environment}. Entorno de ejecuci\'on requerido para ejecutar aplicaciones escritas en Java.
\item[KVM] \texttt{Kernel-based Virtual Machine}. Soluci\'on de virtualizaci\'on completa (o hardware) para Linux sobre arquitecturas x86 que contengan extensiones de virtualizaci\'on (Intel VT-x o AMD-V).
\item[LVM] \texttt{Logical Volume Manager}. Conjunto de aplicativos que permiten de\-fi\-nir una capa de abstracci\'on en el almacenamiento de disco que proporciona mecanismos de asignaci\'on de espacio mucho m\'as flexibles.
\item[MMU] \texttt{Memory Management Unit}. Componente hardware, integrado directamente en el \emph{die} de la CPU desde hace algunas generaciones, encargado de manejar el acceso a memoria por parte de la CPU.
\item[NAS] \texttt{Network Attached Storage}. Dispositivo de almacenamiento que se incorpora a una red existente para incrementar la capacidad de salvaguarda de informaci\'on compartida por los dispositivos de la red.
\item[NAT] \texttt{Network Address Translation}. Proceso que utilizan algunos dis\-po\-si\-ti\-vos de red para modificar el origen y/o destino de la paqueter\'ia en un segmento. Se usa como medio para limitar el n\'umero de direcciones de red repartidas y para ocultar la identidad de los servidores.
\item[NFS] \texttt{Network File System}. Procotolo de un sistema de ficheros distribuido ideado en Sun Microsystems, Inc. para tratar los datos en \'el almacenados como si fuesen locales a los clientes que accediesen a ellos.
\item[NTP] \texttt{Network Time Protocol}. Protocolo dise\~nado para sincronizar la hora actual de los computadores en una red.
\item[OCCI] \texttt{Open Cloud Computing Interface}. Conjunto de especificaciones, re\-par\-ti\-das a trav\'es del \emph{Open Grid Forum}, que definen c\'omo los proveedores de infraestructura como servicio -IaaS- pueden exponer sus ofertas computacionales usando una interfaz unificada.
\item[QCOW2] \texttt{QEMU Copy On Write 2}. Estrategia de optimizaci\'on del almacenamiento en disco que retarda la asignaci\'on de espacio del disco hasta que es absolutamente necesario y formato de la imagen de disco creada por \emph{QEMU} usando este mecanismo.
\item[QEMU] \texttt{Quick EMUlator}. Completa soluci\'on de c\'odigo abierto de emulaci\'on y virtualizaci\'on hardware.
\item[RAID] \texttt{Redudant Array of Independent Disks}. Tecnolog\'ia de al\-ma\-ce\-na\-mien\-to que combina multitud de discos duros convencionales en una unidad l\'ogica mayor, al tiempo que permite definir sobre ella distintas estrategias para mejorar la gesti\'on de la informaci\'on que alberga.
\item[REST] \texttt{Representational State Tranfer}. Subestilo de arquitectura software cliente-servidor para sistemas distribuidos como Internet, en el que se define un protocolo simple, sin estado, cacheable, con interfaz uniforme, etc. para manejar la representaci\'on de cada recurso accesible por los clientes.
\item[RPC] \texttt{Remote Procedure Call}. T\'ecnica de comunicaci\'on entre procesos que permite que un programa de computador provoque la ejecuci\'on de un procedimiento o subrutina en otro espacio de direcciones (incluso en otro computador accesible a trav\'es de una red), sin que el programador se preocupe por escribir los detalles de esta interacci\'on remota.
\item[RPM] \texttt{RPM Package Manager}. Sistema de gesti\'on de paqueter\'ia para Red Hat EL y derivados, capaz, entre otras, de instalar, desinstalar, verificar y actualizar paquetes software \emph{rpm}.
\item[rsync] Software y protocolo de red relacionado que sincroniza ficheros y carpetas entre dos puntos de una red.
\item[Amazon S3] \texttt{Simple Storage Service}. Servicio de almacenamiento es\-ca\-la\-ble, seguro, r\'apido y econ\'omico accesible desde Internet.
\item[SAN] \texttt{Storage Area Network}. Conjunto de dispositivos de almacenamiento de cualquier \'indole conectados para formar una red dedicada exclusivamente a gestionar el volumen de informaci\'on que fluya por ella.
\item[Sandbox] Entorno de ejecuci\'on controlado en el que se encierra un proceso para limitar el da\~no que pudiera ocasionar su mal funcionamiento al sistema.
\item[SELinux] \texttt{Security-Enhanced Linux}. Conjunto de modificaciones del kernel Linux y herramientas en el espacio usuario para crear un \emph{sandbox} de seguridad reforzada, definiendo pol\'iticas de autorizaci\'on de uso de recursos para cada proceso del sistema.
\item[SimpleHTTPServer] M\'odulo de la librer\'ia est\'andar de Python que maneja las peticiones GET y HEAD del protocolo HTTP exponiendo de forma sencilla el contenido de una ruta del servidor.
\item[SPOF] \texttt{Single Point Of Failure}. Se dice de aquel punto singularmente delicado de un sistema, cuyo mal funcionamiento provocar\'ia la ca\'ida del sistema completo.
\item[Amazon SQS] \texttt{Simple Queue Service}. Servicio de cola de mensajes que proporciona Amazon para gestionar la comunicaci\'on as\'incrona entre instancias de su EC2.
\item[SSH] \texttt{Secure SHell}. Protocolo criptogr\'afico de red para establecer comunicaciones seguras entre dos puntos conectados a una red.
\item[sudoers] Fichero de un sistema Unix que contiene la lista de usuarios con habilidad de ejecutar comandos como si fuesen el superusuario.
\item[VMM] \texttt{Virtual Machine Manager}. Aplicaci\'on general para la gesti\'on de m\'aquinas virtuales.
\item[VT-x] Conjunto de extensiones de virtualizaci\'on soportado por una serie de CPUs modernas de Intel, Inc., que abren la posibilidad de explotar la virtualizaci\'on nativa sobre la infraestructura comandada por esa CPU.
\item[yum] \texttt{Yellowdog Updater Modified}. Sistema interactivo para manejar las operaciones de administraci\'on de paqueter\'ia de una instalaci\'on de Red Hat EL o derivados.
\end{description}

