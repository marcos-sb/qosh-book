\chapter*{Summary}

\noindent The history of computation has seen how the unending evolution of technology has promoted changes in its ways and means. Today, \textit{tablets} and \textit{smartphones}, quantitatively inferior managing and memorizing numbers, camp freely in a global market saturated with options. The tendency is clear: users will get to use more than one device to access the Internet and will want to have all their data synchronized and at hand, all the time.

But that is only a part of the equation. At the other side of every service request there lays a server that must deal with an ever increasing troubling traffic volume, while it maintains response delivery in outstanding delay times --- low latency \emph{may} have helped the infant Google rise above the competition. If we also added that the idea of surrounding every implementation effort with energetic efficiency is a transcendental requisite and not simply a good practice, we would have the perfect setup for the proliferation of new distributed paradigms like \emph{Cloud Computing}. Cloud Computing --- or just \emph{the cloud} --- is not intrinsically a new idea but an abstracting on an old idea: \emph{Virtualization}. The clouds' cornerstone is flexibility in providing a client with computational infrastructure.

Another technology that is constantly making headlines is \emph{MapReduce}. If the cloud centers around easing infrastructure exploitation, MapReduce's core strength lies in its speeding up driving large masses of unstructured data; which, together, form an extraordinary computational tandem. The present text will put forth a solution that allows for drawing on computational resources available, exploiting the cooperation of both technologies. Special emphasis has been placed in flexibility of access, being a web browser the only application required to use the service; in simplifying the virtual cluster configuration, by including a self-managed minimum deployment; and in transparency and extensibility, by freeing source code and documentation as open-source software, favoring its usage as starting point for larger installations.


\section*{Keywords}
\noindent Distributed Computing, Virtualization, Cloud Computing, MapReduce, OpenStack, Hadoop.
