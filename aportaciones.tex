\chapter*{Conclusiones}\label{cap:aportaciones}
\markboth{Conclusiones}{Conclusiones}
\addcontentsline{toc}{chapter}{Conclusiones}
\noindent A continuaci\'on se comentan las aportaciones del proyecto ---que hemos denominado \texttt{quick-openstaked-hadoop} o \texttt{qosh}--- y posibles l\'ineas de de\-sa\-rro\-llo para versiones futuras.

\section*{Principales aportaciones}\label{sec:bondadesdeficiencias}
\noindent Se ha dise\~nado e implementado una soluci\'on que reduce considerablemente la complejidad y el coste inherentes a la explotaci\'on el\'astica de infraestructura para ejecutar algoritmia MapReduce. La soluci\'on configura con autonom\'ia el cloud IaaS OpenStack Folsom y gestiona el despliegue bajo demanda del cl\'uster virtual Hadoop encargado del procesamiento MapReduce, permitiendo enviar trabajos y recoger los resultados a trav\'es de su sencilla interfaz web. Las principales caracter\'isticas de la soluci\'on propuesta son:

\begin{description}
    \item[Sencillez] tanto de instalaci\'on como de explotaci\'on. El proceso instalador no puede ser m\'as simple y la interfaz web proporciona los medios para definir despligues Hadoop indeterminadamente grandes, enviar trabajos MapReduce y recoger resultados de forma directa e intuitiva. El Ap\'endice \ref{cap:guiainstalacion} muestra una gu\'ia r\'apida para realizar un despliegue de prueba.
    \item[Integraci\'on vertical] de todos los componentes que posibilitan el fun\-cio\-na\-mien\-to de qosh. El script de instalaci\'on de qosh configura autom\'aticamente el entorno completo de ejecuci\'on en todos los niveles, sin necesidad de interacci\'on por parte del usuario.
    \item[Alto rendimiento] en el despliegue inicial. Qosh se ha concebido desde el principio para ejecutar flujos de trabajo MapReduce y reducir la dificultad de acceso a los cloud. Otras soluciones que facilitan la exploraci\'on de estas tecnolog\'ias ---como \emph{DevStack}---, no ofrecen entornos viables para el desarrollo de aplicaciones reales sobre un cloud debido a su li\-mi\-ta\-do rendimiento.
    \item[Reutilizaci\'on] de los componentes fundamentales. El no haber alterado el modo en que la m\'aquina virtual Hadoop se configura ---siguiendo las directrices de los AWS: usando los \emph{metadatos} inyectados en las instancias al arrancar---, permite que se pueda utilizar libremente en \emph{cualquier} cloud IaaS com\-pa\-ti\-ble. Adicionalmente, se podr\'ia efec\-tuar un r\'apido despliegue de esta imagen sobre un cl\'uster real, es decir, sin cloud. Para este segundo caso habr\'ia que modificar la mec\'anica de ejecuci\'on ge\-ne\-ral de qosh, pero pudiendo beneficiarnos de \emph{todas} las funciones definidas en \texttt{fabfile.py} sin alteraci\'on.
    \item[Adaptaci\'on] de qosh para manejar infraestructura de otros cloud. Usando el fichero \texttt{compute.py} como partida, bastar\'ia reescribir las funciones en \'el definidas, respetando los formatos, para habilitar esa adaptaci\'on.
    \item[Transparencia] completa. El c\'odigo fuente de \emph{todos} los m\'odulos es p\'ublicamente accesible; incluyendo OpenStack, Django, Fabric, Hadoop, Py\-thon y qosh.
    %\item[Cierto acoplamiento] de los m\'odulos de qosh. El comportamiento de cada subsistema ---el cliente de acceso al API de OpenStack (\texttt{compute.py}) o el m\'odulo de control de despliegue virtual (\texttt{fabfile.py})--- se ha codificado directamente sin abstraer \emph{Interfaces}, \emph{Fachadas} y \emph{Delegados} propios de aplicaciones de mayor calado. Adem\'as, las acciones est\'an escritas en las clases que controlan la \emph{vista} en la interfaz web, lo que hace imposible desligar el procesamiento MapReduce de la interfaz web.
    %\item[No soporta clouds h\'ibridos] para tomar provisiones de infraestructura. En parte debido al acoplamiento mencionado en el punto precedente, lo cierto es que \texttt{compute.py} se ha dise\~nado espec\'ificamente para manejar un solo cloud.
    %\item[Almacenamiento inseguro] tanto de los valores de las tareas intermedias MapReduce como de los resultados de los trabajos completos, al residir en el sistema de ficheros local del controlador del cloud. Es decir, a menos que se defina alguna pol\'itica de salvaguarda de informaci\'on a mayores, la fragilidad de los ficheros que contienen los resultados es la misma que la del propio sistema operativo anfitri\'on.
    %\item[Sin control de carga] o restricciones de procesamiento. Los procesos que se encargan de dirigir las ejecuciones se crean bajo demanda sin asignaci\'on de prioridad, posibilidad de pausa o limitaci\'on, lo que podr\'ia saturar con facilidad la capacidad del cloud.
\end{description}

\section*{Trabajo futuro}\label{sec:directricesfuturo}
\noindent Uno de los puntos d\'ebiles de qosh es el cierto grado de acoplamiento existente entre los m\'odulos. Ser\'ia interesante abstraer en una interfaz el cliente de acceso al API REST de los cloud, de forma que se pudiesen implementar distintos \emph{delegados} que adaptasen los mensajes a cada cloud concreto; dejando abierta la posibilidad de usar clouds h\'ibridos. \newline

Siguiendo esta misma din\'amica, algunos casos de uso como \emph{Ver historial} o \emph{Enviar trabajo}, podr\'ian convertirse en objetos, desvincularse de la interfaz web y ser ejecutados en un procesador de acciones. Habiendo desligado estas acciones de la interfaz, ser\'ia sencillo implementar un API REST propio que permitiese a clientes de cualquier tipo ejecutar flujos MapReduce. Este procesador de acciones ser\'ia, por ejemplo, un \emph{pool} de procesos que consumir\'ia las acciones-objeto que estuviesen a la espera en una cola as\'incrona ---como \texttt{Qpid}---, facilitando la escalabilidad horizontal de qosh y la distribuci\'on de carga. 


