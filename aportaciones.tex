\chapter*{Conclusion}\label{cap:aportaciones}
\markboth{Conclusion}{Conclusion}
\addcontentsline{toc}{chapter}{Conclusion}
\noindent This section discusses a series of qosh's contributions and possible lines of development for the future.

\section*{Main Contributions}\label{sec:bondadesdeficiencias}
\noindent qosh has been designed and implemented to be the solution that will considerably reduce the complexity and cost inherent to elastic MapReduce computations. qosh automatically configures a minimum OpenStack Folsom installation before being able to drive a virtual Hadoop cluster deployed atop. As a convenient feature, qosh is also accompanied with a web interface to setup MapReduce jobs and fetch results as they finish execution.

The main qosh characteristics may be summarized as follows.

\begin{description}
    \item[Simplicity:] both for installation and exploitation. The installer cannot be any easier and the web interface provides the means to define Hadoop clusters indeterminately large, send in MapReduce workflows and retrieve results directly and intuitively. The Annex \ref{cap:guiainstalacion} shows a quick guide to complete a testing deployment.
    \item[Vertical integration:] of every component that allows qosh function. The installation script sets up the execution environment for every level needing no interaction from the user.
    \item[High performance:] in the initial deployment. qosh has devised to be simple and flexible but also performing. Other simple solutions like \emph{DevStack}, while easy to configure and run, impose a serious penalty to performance rendering them unable to provide the required environment to develop applications on top of OpenStack --- which, of course, it is not its main purpose.
    \item[Reusability:] of fundamental components. By having followed AWS' guidelines of letting the injected meta-data configure the instances when booting, it allows any IaaS Cloud to be used --- provided it be AWS-compatible ---, or seen from another perspective, it allows the default Hadoop VM to be installed directly on a real cluster, i.e., without virtualization.
    \item[Adaptability:] of qosh's to handle infrastructure from other clouds. Using the Compute module as starting point, a partial rewrite would suffice to support any IaaS Cloud.
    \item[Transparency:] of the whole. The code of any module may be openly accessed, downloaded and modified --- some restrictions may apply ---; including OpenStack, Django, Fabric, Hadoop, Python and qosh.
\end{description}

\section*{Future Development}\label{sec:directricesfuturo}
\noindent One of qosh's weak points lies in the coupling among the main modules. It would be interesting to abstract an interface with the cloud's REST API access client, so that different delegates could implement it and adapt the messages to the particular cloud syntax, opening indirectly the possibility to hybrid cloud deployments.

Following those dynamics, some use cases like \emph{Show History} or \emph{Send Job}, could become objects, decouple from the web interface and be executed in an action processor object. Having unlinked those actions off the interface, it would not be hard to implement a particular REST API that would allow clients of any nature to execute MapReduce workflows. This action processor object, might be designed as a thread pool that would consume object-actions fed from the interface controller, facilitating scaling out and load balancing.
