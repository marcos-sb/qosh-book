\chapter{Glossary of Terms}\label{cap:glosario}
\begin{description}
\item[ACPI] \texttt{Advanced Configuration and Power Interface}. A power management specification that makes hardware status information available to the operating system.
\item[AMD-V] \texttt{Advanced Micro Devices Virtualization}. Set of CPU extensions implemented by Advanced Micro Devices, Inc. to support Full Hardware Virtualization.
\item[API] \texttt{Application Programming Interface}. Protocol definition interface between two or more communicating entities. It specifies an abstract format by which collaborating entities can interchange information.
\item[APIC] \texttt{Advanced Programmable Interrupt Controller}. Family of interrupt controllers used to combine different interruption signal sources into one or more CPU lines, allowing for the assignment of priority levels to each kind of interruption.
\item[AWS] \texttt{Amazon Web Services}. Set of web services offered by Amazon Inc. that let a user exploit, remotely and on-demand, Amazon's computational infrastructure.
\item[Hard coding] In Software Engineering, \emph{Hard coding} refers to the bad practice of including an input or configuration value within the source code that controls the application logic.
\item[CLI] \texttt{Command Line Interface}. Refers to every application whose main interface is the terminal or command line.
\item[CUDA] \texttt{Compute Unified Device Architecture}. Parallel computing platform and programming model devised at NVIDIA and implemented in its GPUs.
\item[DFS] \texttt{Distributed File System}. Any kind of file system stored over a network of computing devices and concurrently accessible by multiples users in different points.
\item[DHCP] \texttt{Dynamic Host Configuration Protocol}. Network protocol used to automatically and dynamically configure access parameters to a network.
\item[DNS] \texttt{Domain Name Server}. Name service that maps certain device information from one domain to another. It is typically used to locate resources in a network by translating their names to their IP addresses.
\item[Amazon EC2] \texttt{Elastic Compute Cloud}. Amazon, Inc.'s web service allowing for on-demand provisioning of computational infrastructure.
\item[EPT] \texttt{Extended Page Tables}. Intel's hardware virtualization system permitting a guest operating system to directly modify its own memory table pages, effectively requiring no VMM override to handle page misses.
\item[Hash Function] An algorithm that associates a larger variable-length data space with a smaller fixed-length data space.
\item[HTTP] \texttt{HyperText Transfer Protocol}. Application protocol for distributed, collaborative, hypermedia information systems. HTTP is the foundation of data communication for the World Wide Web.
\item[HVM] \texttt{Hardware Virtual Machine}. Virtualization approach by which a modified CPU is able to create a virtual domain for applications to exploit infrastructure resources with reduced penalty on performance. Also called \emph{Native Virtualization}.
\item[iptables] Administrative tool and associated Linux service that allows configuring the routing tables provided by the kernel firewall and the chains and rules it stores.
\item[IT] \texttt{Information Technology}. It comprises any concept related to information and the technology to handle it, like networks, hardware, software, the Internet, as well as the people working with these technologies.
\item[JRE] \texttt{Java Runtime Environment}. Environment that need be installed in any computer requiring the execution of Java applications locally.
\item[KVM] \texttt{Kernel-based Virtual Machine}. Full Virtualization solution for the Linux kernel that turns it into a hypervisor. KVM requires a processor with hardware virtualization extensions.
\item[LVM] \texttt{Logical Volume Manager}. Tool set that allows a user to define an storage abstraction layer on top of conventional persistence that is more flexible than conventional partitioning schemes.
\item[MMU] \texttt{Memory Management Unit}. Hardware device, on-die in modern CPU architectures, that handles access to main memory.
\item[NAS] \texttt{Network Attached Storage}. Storage device that is attached to a preexisting network to increase the shared storage space.
\item[NAT] \texttt{Network Address Translation}. Process by which some networking devices rewrite the source and/or destination fields in telecommunications within a network segment. It is used as a means to limit the number of issued network addresses and to conceal the identity of devices behind the NAT.
\item[NFS] \texttt{Network File System}. A file system developed by Sun Microsystems, Inc. implementing client/server model that allows users to access files across a network and treat them as if they resided in a local file directory.
\item[NTP] \texttt{Network Time Protocol}. It is a networking protocol for clock synchronization between computer systems over packet-switched, variable-latency data networks.
\item[OCCI] \texttt{Open Cloud Computing Interface}. A set of specifications delivered through the Open Grid Forum, for cloud computing service providers to offer their services defining a unified interface.
\item[QCOW2] \texttt{QEMU Copy On Write 2}. Optimization strategy by which a storage device will delay the allocation of storage space until it is actually needed.
\item[QEMU] \texttt{Quick EMUlator}. Free and open source hosted hypervisor that performs hardware virtualization.
\item[RAID] \texttt{Redudant Array of Independent Disks}. It is a data storage virtualization technology that combines multiple disk drive components into a logical unit for the purposes of data redundancy or performance improvement.
\item[REST] \texttt{Representational State Tranfer}. Client/server architectural style that defines a simple, stateless, cacheable, interface-uniform protocol used to handle resource representation in distributed systems.
\item[RPC] \texttt{Remote Procedure Call}. An inter-process communication that allows a computer program to cause a subroutine or procedure to execute in another address space (commonly on another computer on a shared network) without the programmer explicitly coding the details for this remote interaction.
\item[RPM] \texttt{RPM Package Manager}. A package management system for the Red Hat Enterprise Linux and derivatives.
\item[rsync] Is a file synchronization and file transfer program for Unix-like systems that minimizes network data transfer by using a form of delta encoding called the rsync algorithm. rsync can compress the data transferred further using zlib compression, and SSH or stunnel can be used to encrypt the transfer.
\item[Amazon S3] \texttt{Simple Storage Service}. Scalable, safe, fast and inexpensive storage service consumed through the Internet and provided by Amazon, Inc.
\item[SAN] \texttt{Storage Area Network}. It is a dedicated network that provides access to consolidated, block level data storage.
\item[Sandbox] In the context of software development and revision control, it is a testing environment that isolates untested code changes and outright experimentation from the production environment or repository respectively.
\item[SELinux] \texttt{Security-Enhanced Linux}. A Linux kernel security module that provides a mechanism for supporting access control security policies, including United States Department of Defense-style mandatory access controls (\emph{MAC}).
\item[SPOF] \texttt{Single Point Of Failure}. In Systems Engineering it is the point of the architecture that if it failed it would prevent the whole system from functioning.
\item[Amazon SQS] \texttt{Simple Queue Service}. Queue service provided by Amazon, Inc. to handle asynchronous message passing between EC2 instances.
\item[SSH] \texttt{Secure SHell}. A cryptographic network protocol for secure data communication, remote command-line login, remote command execution, and other secure network services between two networked computers.
\item[sudoers] The file in a *nix system that specifies which users can execute commands as if they were the root user.
\item[VMM] \texttt{Virtual Machine Manager}. A management solution for the virtualized data center. It can be used to configure the virtualization host, networking, and storage resources, in order to create and deploy virtual machines and services to private clouds.
\item[VT-x] Intel, Inc.'s CPU extension set allowing for full hardware virtualization.
\item[yum] \texttt{Yellowdog Updater Modified}. An open source command-line package-management utility for Linux operating systems using the RPM Package Manager.
\end{description}

